\documentclass[14pt]{extarticle}
\usepackage[spanish]{babel}
\usepackage[utf8]{inputenc}
\usepackage{enumerate}
\usepackage{xparse, amsmath, physics}
\usepackage{cancel}

\decimalpoint

\usepackage{anysize}
\marginsize{1cm}{1cm}{1.3cm}{1.3cm}
\renewcommand{\baselinestretch}{1.2}
\parindent  = 0mm
\parskip = 4mm

\usepackage[usenames]{color}
\definecolor{azul}{RGB}{10,80,190}
\definecolor{negro}{RGB}{0,0,0}
\definecolor{rojo}{RGB}{190,80,10}
\definecolor{verde}{RGB}{0,120,50}

\begin{document}
    \title{
        Matemáticas para las ciencias aplicadas IV\\
        Tarea 1
    }
    \author{
        Careaga Carrillo Juan Manuel\\
        Quiróz Castañeda Edgar\\
        Soto Corderi Sandra del Mar
    }
    \date{13 de marzo de 2019}
    \maketitle
    \thispagestyle{empty}
    \begin{enumerate}
        % Ejercicio 1
        \pagebreak
        \item {
            Resolver las ecuaciones diferenciales
            \begin{enumerate}
                % a)
                \item {
                    $\dv{y}{t}+\sqrt{1+t^2}e^{-t}y=0$
                    con $y(0)=1$

                    \color{azul}
                    % Respuesta
                }
                
                % b)
                \item {
                    $\dv{y}{t}+\frac{t}{1-t^2}y=1-\frac{t^3}{1+t^4}y$.
                    Graficar algunas soluciones.

                    \color{azul}
                    % Respuesta
                }
                
                % c)
                \item {
                    $\cos{y}\dv{y}{t}=\frac{-t\sen{y}}{1+t^2}$
                    con $y(1)=\frac{\pi}{2}$

                    \color{azul}
                    % Respuesta
                }
                
                % d)
                \item {
                    $\dv{y}{t}=1-t+y^2-ty^2$.
                    Graficar algunas soluciones.

                    \color{azul}
                    % Respuesta
                }
                
                % e)
                \item {
                    $ye^{xy}\cos{2x}-2e^{xy}\sen{2x}+2x+
                    (xe^{xy}\cos{2x}-3)\dv{y}{x}=0$

                    \color{azul}
                    % Respuesta
                    Consideremos $M = ye^{xy}\cos{2x}-2e^{xy}\sen{2x}+2x$ y $N = 
                    xe^{xy}\cos{2x}-3$. \\
                    Entoces
                    \begin{align*}
                        \dv{M}{y} 
                        &= \dv{(ye^{xy}\cos{2x}-2e^{xy}\sen{2x}+2x)}{y}
                        = \dv{ye^{xy}}{y}\cos{2x} - 2\dv{e^{xy}}{y}\sen{2x} + 0 \\
                        &= (e^{xy} + xye^{xy}) \cos{2x}- 2xye^{xy}\sen{2x} \\
                        &= e^{xy}\cos{2x} + xye^{xy}\cos{2x} - 2xye^{xy}\sen{2x}
                    \end{align*}
                    Y
                    \begin{align*}
                        \dv{N}{x} 
                        &= \dv{(xe^{xy}\cos{2x}-3)}{x} = \dv{xe^{xy}\cos{2x}}{x}
                        -\dv{3}{x} \\
                        &= e^{xy}\cos{2x} + x (ye^{xy}\cos{2x} 
                        + e^{xy}(-\sen{2x})2) + 0 \\
                        &= e^{xy}\cos{2x} + x(ye^{xy}\cos{2x} - 2ye^{xy}\sen{2x})\\
                        &= e^{xy}\cos{2x} + xye^{xy}\cos{2x} - 2xye^{xy}\sen{2x}\\
                        &= \dv{M}{y} 
                    \end{align*}
                    Por lo que la ecuación es exacta. Esto es, que existe una 
                    $\phi(x, y)$ tal que $\dv{\phi}{x} = M$ y $\dv{\phi}{y} = N$\\
                    Para encontrar la primitiva, es más sencillo integrar $N$
                    respecto a $y$.
                    \begin{align*}
                        \phi(x, y) &= \int{N dy}
                        = \int{(xe^{xy}\cos{2x}-3) dy} \\
                        &= \cos{2x} \int{xe^{xy} dy} - \int{3dy}
                        = e^{xy}\cos{2x} - 3y + g(x)
                    \end{align*}
                    Y queda como incógnita $g(x)$, pero se puede obtener usando 
                    el hecho de que $M$ es la derivada respecto a $x$ de la 
                    primitva
                    \begin{align*}
                        M 
                        &= \dv{\phi(x, y)}{x} 
                        = \dv{(e^{xy}\cos{2x} - 3y + g(x))}{x} \\
                        ye^{xy}\cos{2x} - 2e^{xy}\sen{2x} + 2x 
                        &= ye^{xy}\cos{2x} - 2e^{xy}\sen{2x} + g'(x)\\
                        \implies g'(x) = 2x &\implies g(x) = \int{2xdx} = x^{2}
                    \end{align*}
                    Por lo que la primitiva sin incógnitas es 
                    \[\phi(x, y) = e^{xy}\cos{2x} - 3y + 2x\]
                    Luego, despejando $y$.
                    \begin{align*} 
                        \implies e^{xy}\cos{2x} &= 3y - 2x \\
                        \implies e^{xy} &= \frac{3y - 2x}{\cos{2x}} \\
                        \implies xy &= ln(\frac{3y - 2x}{\cos{2x}}) \\
                        \implies y &= \frac{1}{x}ln(\frac{3y - 2x}{\cos{2x}}) \\
                        \implies y &= ln((\frac{3y - 2x}{\cos{2x}})^{x^{-1}})
                    \end{align*}
                    Vemos que no se puede despejar, pero se pueden aproximar las 
                    soluciones.
                }
                
                % f)
                \item {
                    $(t+2)\sen{y}+t\cos{y}\dv{y}{t}=0$

                    \color{azul}
                    % Respuesta
                    Modificando la ecuación, tenemos que 
                    \begin{align*}
                        &\implies t\cos{y}\dv{y}{t} = -(t+2)\sen{y} \\
                        &\implies \dv{y}{t} = -\frac{(t+2)\sen{y}}{t\cos{y}} \\
                        &\implies \dv{y}{t} = -\frac{\frac{t+2}{t}}{\cot{y}}
                    \end{align*}
                    Que es de la forma de una ecuación separable. Por lo que, 
                    para resolverla, hay que resolver
                    \begin{align*}
                        &\implies \cot{y} dy = -(1 + \frac{2}{t})dt \\
                        &\implies \int{\cot{y} dy} = -\int{(1 + \frac{2}{t})dt} \\
                        &\implies ln \abs{\sin{y}}= -(t + 2 ln \abs{t}) + C \\
                        &\implies \abs{\sin{y}} = C e^{-(t + 2 ln \abs{t})} \\
                        &\implies \sin{y} = \pm (C e^{-(t + 2 ln \abs{t})}) \\
                        &\implies y = \arcsin (\pm (C e^{-(t + 2 ln \abs{t})}))
                    \end{align*}
                }
                
                % g)
                \item {
                    $\left(3t+\frac{6}{y}\right)
                    +\left(\frac{t^2}{y}+3\frac{y}{t}\right)\dv{y}{t}=0$

                    \color{azul}
                    % Respuesta
                    Primero, hay que multiplicar todo por $ty$.
                    \[(3t^{2}y+6t)+(t^{3}+3y^{2})\dv{y}{t}=0\]
                    Entonces, con $M = 3t^{2}y+6t$ y $N = t^{3}+3y^{2}$ se tiene
                    que 
                    \[\dv{M}{y} = 3t^{2} = \dv{N}{t}\]
                    Por lo que la ecuación es exacta, esto es que existe una 
                    $\phi$ primitiva de $M$ y $N$\\
                    Luego, integrando $M$ respecto a $t$.
                    \[\int{M dt} = \int{(3t^{2}y+6t)dt} = t^{3}y + 3t^{2} + g(y)\]
                    Integrando $N$ respecto a $y$.
                    \[\int{N dy} = \int{(t^{3}+3y^{2})dt} = t^{3}y + y^{3} + h(t)\]
                    Por lo que la forma de la expresión sin incógnitas es 
                    \[\phi(y, t) = t^{3}y + y^{3} + 3t^{2}\]
                    Luego, despejando $y$
                    \begin{align*}
                        &\implies y = -\frac{y^{3} + 3t^{2}}{t^{3}} \\
                        &\implies y + \frac{y^{3}}{t^{3}} = \frac{3}{t} \\
                        &\implies y + \frac{y^{3}}{t^{3}} - \frac{3}{t}  = 0 \\
                        &\implies y^{3} + yt^{3} - 3t^{2}  = 0
                    \end{align*}
                    Que está en la forma de una ecuación cúbica deprimida que, 
                    por la fórmula de Cardano, tiene solución real
                    \begin{align*}
                        y &= \sqrt[3]{-\frac{-3t^{2}}{2} + \sqrt{\frac{(-3t^{2})^{2}}{4} 
                    + \frac{(t^{3})^{3}}{27}}} + \sqrt[3]{-\frac{-3t^{2}}{2} - \sqrt{\frac{(-3t^{2})^{2}}{4} 
                    + \frac{(t^{3})^{3}}{27}}} \\
                    &= \sqrt[3]{\frac{3t^{2}}{2} + \sqrt{\frac{9t^{4}}{4} 
                    + \frac{t^{9}}{27}}} + \sqrt[3]{\frac{3t^{2}}{2} - \sqrt{\frac{9t^{4}}{4} 
                    + \frac{t^{9}}{27}}}
                    \end{align*}
                }
            \end{enumerate}
        }
        
        % Ejercicio 2
        \pagebreak
        \item {
            Hallar todas las funciones $g(t)$ que hacen que la ecuación
            diferencial $$y^2\sen{t}+yg(t)\dv{y}{t}=0$$ sea exacta.

            \color{azul}
            Supongamos que la ecuación dada es exacta.
            
            Sea $M(t,y)=y^2\sen{t}$ y sea $N(t,y)=yg(t)$. Como la ecuación
            diferencial es exacta, entonces se cumple que
            $\pdv{M}{y}=\pdv{N}{t}$, es decir
            $$\pdv{M}{y}=2y\sen{t}=yg'(t)=\pdv{N}{t}$$
            Entonces
            $$g'(t)=2\sen{t}$$
            $$g(t)=\int{2\sen{t}\dd t}=-2\cos{t}+h(y)$$
            Para que la ecuación diferencial pueda resolverse es necesario que
            la función $g$ sólo dependa de una variable, ya estaba establecida
            la dependencia con la variable $t$, por lo que $h(y)=C$.

            Por lo que la familia de funciones $g(t)=-2\cos{t}+C$ hacen que la
            ecuación diferencial sea exacta.
        }
        
        % Ejercicio 3
        \pagebreak
        \item {
            Las ecuaciones diferenciales de la forma $$\dv{y}{t}=f(y/t)$$ se
            pueden resolver si se hace el cambio de variable $v=y/t$. Mostrar
            que la ecuación toma la forma $$t\dv{v}{t}+v=f(v).$$ Usar este
            método para resolver $$\dv{y}{t}=\frac{t+y}{t-y}.$$

            \color{azul}
            Sea $v=y/t$, entonces $y=vt$ y podemos hacer un cambio de variable
            \begin{align*}
                \dv{y}{t}             &= f(y/t)\\[.3cm]
                \dv{(vt)}{t}          &= f(v)\\[.3cm]
                v\dv{t}{t}+t\dv{v}{t} &= f(v)\\[.3cm]
                t\dv{v}{t}+v          &= f(v)
            \end{align*}
            Ahora, para resolver la ecuación $\dv{y}{t}=\frac{t+y}{t-y}$
            hacemos el cambio de variable $v=y/t$, o bien, $y=vt$.
            \begin{align*}
                \dv{(vt)}{t}          &= \frac{1+vt}{1-vt}\\[.3cm]
                v\dv{t}{t}+t\dv{v}{t} &= \frac{t(1+v)}{t(1-v)}\\[.3cm]
                v+tv'                 &= \frac{1+v}{1-v}\\[.3cm]
                tv'                   &= \frac{1+v}{1-v}-v\\[.3cm]
                tv'                   &= \frac{1+v^2}{1-v}\\[.3cm]
                \frac{1-v}{1+v^2}v'   &= \frac{1}{t}
            \end{align*}
            Sea $F(v)$ tal que $F'(v)=\frac{1-v}{1+v^2}$ entonces $\dv{F(v(t))}
            {t}=\dv{F}{v}\dv{v}{t}=\frac{1-v}{1+v^2}v'$, integramos respecto de
            $t$ de ambos lados
            \begin{align*}
                \int{\dv{F(v)}{t}\,\dd t} &= \int{\frac{1}{t}\,\dd t}\\[.3cm]
                \int{\frac{1-v}{1+v^2}\,\dd v} &= \ln{\abs{t}}+C\\[.3cm]
                \int{\frac{1}{1+v^2}\,\dd v}-\int{\frac{v}{1+v^2}\,\dd v}
                &= \ln{\abs{t}}+C\\[.3cm]
                \tan^{-1}{v}-\frac{1}{2}\ln{\abs{1+v^2}}&=\ln{\abs{t}}+C
            \end{align*}
            No es posible despejar a la $v$, por lo que sólo resta revertir el
            cambio de variable
            $$\tan^{-1}{\left(\frac{y}{t}\right)}
            -\frac{1}{2}\ln{\abs{1+\left(\frac{y}{t}\right)^2}}
            =\ln{\abs{t}}+C$$
        }
        
        % Ejercicio 4
        \pagebreak
        \item {
            Una población crece de acuerdo a la ley logística, y tiene un
            límite de $5\times 10^8$ individuos. Cuando la población es baja se
            duplica cada 40 minutos. ¿Qué valor tendrá la población después de
            dos horas si inicialmente era de \emph{a}) $10^8$ individuos y
            \emph{b}) $10^9$
            individuos?

            \color{azul}
            La ley logística nos dice que
            \[
                P(t)=\frac{P_0 a}{P_0 b+(a-P_0 b)e^{-a(t-t_0)}}
            \]
            donde $\frac{a}{b}=5\times 10^8$ marca el límite o estabilidad
            poblacional, $P_0$ es la población inicial, $t_0$ es el tiempo
            inicial (que en este caso lo consideraremos $t_0=0$).

            Cuando la población es baja utilizamos la ecuación logística
            $\dv{P}{t}=KP_0$ cuya solución es $P(t)=(P_0)e^{at}$. Utilizaremos
            esta información para encontrar el valor de $a$ y eventualmente el
            de $b$.
            \begin{enumerate}
                \item Si $P_0=10^8$ y sabemos que
                $P(40)=2\times 10^8=10^8e^{a(40)}$. Despejando a la $a$
                \begin{align*}
                    2 &= e^{40a}\\
                    \ln{2} &= 40a\\
                    a &= \frac{\ln{2}}{40}\\
                    a &\approx 0.01733
                \end{align*}
                Por el límite de población, tenemos que
                $\frac{a}{b}=5\times 10^8$, ahora que tenemos $a$ podemos
                encontrar $b$
                \begin{align*}
                    b &= \frac{a}{5\times 10^8}\\[.2cm]
                    b &= \frac{\frac{\ln{2}}{40}}{5\times 10^8}\\[.2cm]
                    b &= \frac{\ln{2}}{200\times 10^8}\\[.2cm]
                    b &\approx 34.65\times 10^{-12}
                \end{align*}
                Ya que tenemos todo lo que necesitamos, podemos aplicar la
                ecuación logística inicial, con el tiempo en minutos (para
                simplificar la notación se han dejado los valores de $a$ y de
                $b$ expresado con sus literales):
                \begin{align*}
                    P(120) &= \frac{a\times 10^8}
                    {b\times 10^8+(a-b\times 10^8)e^{-a(120-0)}}\\[.3cm]
                    &\approx \frac{1732867.95}{0.003465+(0.01386)
                    e^{-0.01733(120)}}\\
                    &\approx 333'333'333.33\\
                    &\approx 3.3333\times 10^8 \text{ individuos.}
                \end{align*}

                \item Ahora si $P_0=10^9$ individuos procedemos de manera
                análoga.
                \begin{align*}
                    P(40)=2\times\cancel{10^9}&=\cancel{10^9}e^{a(40)}\\[.2cm]
                    2 &= e^{40a}
                \end{align*}
                Por lo que los valores de $a$ y de $b$ no dependen de la
                población inicial y son los mismos que en inciso anterior.

                Aplicando la ecuación logística
                \begin{align*}
                    P(120) &= \frac{a\times 10^9}
                    {b\times 10^9+(a-b\times 10^9)e^{-a(120-0)}}\\[.3cm]
                    &\approx \frac{17328679.514}{0.03465+(-0.01732)
                    e^{-0.01733(120)}}\\[.3cm]
                    &\approx 533'333'333.33\\[.2cm]
                    &\approx 5.3333\times 10^8 \text{ individuos.}
                \end{align*}
                Notemos que en este caso la población supera el límite
                poblacional que era de $5\times 10^8$, esto puede deberse a que
                consideramos la población inicial de $10^9$ como ``baja''
                cuando en realidad no hay un criterio adecuado para determinar
                cuando la población es considerada como ``baja''. Sin embargo
                al aplicar la ecuación logística para tiempos más grandes
                podemos notar que la población se estabiliza hacia su límite,
                por ejemplo $P(240)\approx5.04\times 10^8$ individuos.
            \end{enumerate}
        }
    \end{enumerate}
\end{document}