\documentclass[14pt]{extarticle}
\usepackage[spanish]{babel}
\usepackage[utf8]{inputenc}
\usepackage{enumerate}
\usepackage{xparse, amsmath, physics}

\usepackage{anysize}
\marginsize{1cm}{1cm}{1.3cm}{1.3cm}
\renewcommand{\baselinestretch}{1.2}
\parindent  = 0mm
\parskip = 4mm

\usepackage[usenames]{color}
\definecolor{azul}{RGB}{10,80,190}
\definecolor{negro}{RGB}{0,0,0}
\definecolor{rojo}{RGB}{190,80,10}
\definecolor{verde}{RGB}{0,120,50}

\begin{document}
    \title{
        Matemáticas para las ciencias aplicadas IV\\
        Tarea 1
    }
    \author{
        Careaga Carrillo Juan Manuel\\
        Quiróz Castañeda Edgar\\
        Soto Corderi Sandra del Mar
    }
    \date{13 de marzo de 2019}
    \maketitle
    \thispagestyle{empty}
    \pagebreak
    \begin{enumerate}
        % Ejercicio 1
        \item {
            Resolver las ecuaciones diferenciales
            \begin{enumerate}
                % a)
                \item {
                    $\dv{y}{t}+\sqrt{1+t^2}e^{-t}y=0$
                    con $y(0)=1$

                    \color{azul}
                    % Respuesta
                }
                
                % b)
                \item {
                    $\dv{y}{t}+\frac{t}{1-t^2}y=1-\frac{t^3}{1+t^4}y$.
                    Graficar algunas soluciones.

                    \color{azul}
                    % Respuesta
                }
                
                % c)
                \item {
                    $\cos{y}\dv{y}{t}=\frac{-t\sen{y}}{1+t^2}$
                    con $y(1)=\frac{\pi}{2}$

                    \color{azul}
                    % Respuesta
                }
                
                % d)
                \item {
                    $\dv{y}{t}=1-t+y^2-ty^2$.
                    Graficar algunas soluciones.

                    \color{azul}
                    % Respuesta
                }
                
                % e)
                \item {
                    $ye^{xy}\cos{2x}-2e^{xy}\sen{2x}+2x+
                    (xe^{xy}\cos{2x}-3)\dv{y}{x}=0$

                    \color{azul}
                    % Respuesta
                }
                
                % f)
                \item {
                    $(t+2)\sen{y}+t\cos{y}\dv{y}{t}=0$

                    \color{azul}
                    % Respuesta
                }
                
                % g)
                \item {
                    $\left(3t+\frac{6}{y}\right)
                    +\left(\frac{t^2}{y}+3\frac{y}{t}\right)\dv{y}{t}=0$

                    \color{azul}
                    % Respuesta
                }
            \end{enumerate}
        }
        
        % Ejercicio 2
        \item {
            Hallar todas las funciones $g(t)$ que hacen que la ecuación
            diferencial $$y^2\sen{t}+yg(t)\dv{y}{t}=0$$ sea exacta.

            \color{azul}
            % Respuesta
        }
        
        % Ejercicio 3
        \item {
            Las ecuaciones diferenciales de la forma $$\dv{y}{t}=f(y/t)$$ se
            pueden resolver si se hace el cambio de variable $v=y/t$. Mostrar
            que la ecuación toma la forma $$t\dv{v}{t}+v=f(v).$$ Usar este
            método para resolver $$\dv{y}{t}=\frac{t+y}{t-y}.$$

            \color{azul}
            % Respuesta
        }
        
        % Ejercicio 4
        \item {
            Una población crece de acuerdo a la ley logística, y tiene un
            límite de $5\times 10^8$ individuos. Cuando la población es baja se
            suplica cada 40 minutos. ¿Qué valor tendrá la población después de
            dos horas si inicialmente era de a) $10^8$ individuos y b) $10^9$
            individuos?

            \color{azul}
            % Respuesta
        }
    \end{enumerate}
\end{document}