\documentclass[14pt]{extarticle}
\usepackage[spanish]{babel}
\usepackage[utf8]{inputenc}
\usepackage{enumerate}
\usepackage{xparse, amsmath, physics}

\usepackage{anysize}
\marginsize{1cm}{1cm}{1.3cm}{1.3cm}
\renewcommand{\baselinestretch}{1.2}
\parindent  = 0mm
\parskip = 4mm

\usepackage[usenames]{color}
\definecolor{azul}{RGB}{10,80,190}
\definecolor{negro}{RGB}{0,0,0}
\definecolor{rojo}{RGB}{190,80,10}
\definecolor{verde}{RGB}{0,120,50}

\begin{document}
    \title{
        Matemáticas para las ciencias aplicadas IV\\
        Tarea 1
    }
    \author{
        Careaga Carrillo Juan Manuel\\
        Quiróz Castañeda Edgar\\
        Soto Corderi Sandra del Mar
    }
    \date{13 de marzo de 2019}
    \maketitle
    \thispagestyle{empty}
    \pagebreak
    \begin{enumerate}
        % Ejercicio 1
        \item {
            Resolver las ecuaciones diferenciales
            \begin{enumerate}
                % a)
                \item {
                    $\dv{y}{t}+\sqrt{1+t^2}e^{-t}y=0$
                    con $y(0)=1$

                    \color{azul}
                    % Respuesta
                }
                
                % b)
                \item {
                    $\dv{y}{t}+\frac{t}{1-t^2}y=1-\frac{t^3}{1+t^4}y$.
                    Graficar algunas soluciones.

                    \color{azul}
                    % Respuesta
                }
                
                % c)
                \item {
                    $\cos{y}\dv{y}{t}=\frac{-t\sen{y}}{1+t^2}$
                    con $y(1)=\frac{\pi}{2}$

                    \color{azul}
                    % Respuesta
                }
                
                % d)
                \item {
                    $\dv{y}{t}=1-t+y^2-ty^2$.
                    Graficar algunas soluciones.

                    \color{azul}
                    % Respuesta
                }
                
                % e)
                \item {
                    $ye^{xy}\cos{2x}-2e^{xy}\sen{2x}+2x+
                    (xe^{xy}\cos{2x}-3)\dv{y}{x}=0$

                    \color{azul}
                    % Respuesta
                    Consideremos $M = ye^{xy}\cos{2x}-2e^{xy}\sen{2x}+2x$ y $N = 
                    xe^{xy}\cos{2x}-3$. \\
                    Entoces
                    \begin{align*}
                        \dv{M}{y} 
                        &= \dv{(ye^{xy}\cos{2x}-2e^{xy}\sen{2x}+2x)}{y}
                        = \dv{ye^{xy}}{y}\cos{2x} - 2\dv{e^{xy}}{y}\sen{2x} + 0 \\
                        &= (e^{xy} + xye^{xy}) \cos{2x}- 2xye^{xy}\sen{2x} \\
                        &= e^{xy}\cos{2x} + xye^{xy}\cos{2x} - 2xye^{xy}\sen{2x}
                    \end{align*}
                    Y
                    \begin{align*}
                        \dv{N}{x} 
                        &= \dv{(xe^{xy}\cos{2x}-3)}{x} = \dv{xe^{xy}\cos{2x}}{x}
                        -\dv{3}{x} \\
                        &= e^{xy}\cos{2x} + x (ye^{xy}\cos{2x} 
                        + e^{xy}(-\sen{2x})2) + 0 \\
                        &= e^{xy}\cos{2x} + x(ye^{xy}\cos{2x} - 2ye^{xy}\sen{2x})\\
                        &= e^{xy}\cos{2x} + xye^{xy}\cos{2x} - 2xye^{xy}\sen{2x}\\
                        &= \dv{M}{y} 
                    \end{align*}
                    Por lo que la ecuación es exacta. Esto es, que existe una 
                    $\phi(x, y)$ tal que $\dv{\phi}{x} = M$ y $\dv{\phi}{y} = N$\\
                    Para encontrar la primitiva, es más sencillo integrar $N$
                    respecto a $y$.
                    \begin{align*}
                        \phi(x, y) &= \int{N dy}
                        = \int{(xe^{xy}\cos{2x}-3) dy} \\
                        &= \cos{2x} \int{xe^{xy} dy} - \int{3dy}
                        = e^{xy}\cos{2x} - 3y + g(x)
                    \end{align*}
                    Y queda como incógnita $g(x)$, pero se puede obtener usando 
                    el hecho de que $M$ es la derivada respecto a $x$ de la 
                    primitva
                    \begin{align*}
                        M 
                        &= \dv{\phi(x, y)}{x} 
                        = \dv{(e^{xy}\cos{2x} - 3y + g(x))}{x} \\
                        ye^{xy}\cos{2x} - 2e^{xy}\sen{2x} + 2x 
                        &= ye^{xy}\cos{2x} - 2e^{xy}\sen{2x} + g'(x)\\
                        \implies g'(x) = 2x &\implies g(x) = \int{2xdx} = x^{2}
                    \end{align*}
                    Por lo que la primitiva sin incógnitas es 
                    \[\phi(x, y) = e^{xy}\cos{2x} - 3y + 2x\]
                    Luego, despejando $y$.
                    \begin{align*} 
                        \implies e^{xy}\cos{2x} &= 3y - 2x \\
                        \implies e^{xy} &= \frac{3y - 2x}{\cos{2x}} \\
                        \implies xy &= ln(\frac{3y - 2x}{\cos{2x}}) \\
                        \implies y &= \frac{1}{x}ln(\frac{3y - 2x}{\cos{2x}}) \\
                        \implies y &= ln((\frac{3y - 2x}{\cos{2x}})^{x^{-1}})
                    \end{align*}
                    Vemos que no se puede despejar, pero se pueden aproximar las 
                    soluciones.
                }
                
                % f)
                \item {
                    $(t+2)\sen{y}+t\cos{y}\dv{y}{t}=0$

                    \color{azul}
                    % Respuesta
                    Modificando la ecuación, tenemos que 
                    \begin{align*}
                        &\implies t\cos{y}\dv{y}{t} = -(t+2)\sen{y} \\
                        &\implies \dv{y}{t} = -\frac{(t+2)\sen{y}}{t\cos{y}} \\
                        &\implies \dv{y}{t} = -\frac{\frac{t+2}{t}}{\cot{y}}
                    \end{align*}
                    Que es de la forma de una ecuación separable. Por lo que, 
                    para resolverla, hay que resolver
                    \begin{align*}
                        &\implies \cot{y} dy = -(1 + \frac{2}{t})dt \\
                        &\implies \int{\cot{y} dy} = -\int{(1 + \frac{2}{t})dt} \\
                        &\implies ln \abs{\sin{y}}= -(t + 2 ln \abs{t}) + C \\
                        &\implies \abs{\sin{y}} = C e^{-(t + 2 ln \abs{t})} \\
                        &\implies \sin{y} = \pm (C e^{-(t + 2 ln \abs{t})}) \\
                        &\implies y = \arcsin (\pm (C e^{-(t + 2 ln \abs{t})}))
                    \end{align*}
                }
                
                % g)
                \item {
                    $\left(3t+\frac{6}{y}\right)
                    +\left(\frac{t^2}{y}+3\frac{y}{t}\right)\dv{y}{t}=0$

                    \color{azul}
                    % Respuesta
                    Primero, hay que multiplicar todo por $ty$.
                    \[(3t^{2}y+6t)+(t^{3}+3y^{2})\dv{y}{t}=0\]
                    Entonces, con $M = 3t^{2}y+6t$ y $N = t^{3}+3y^{2}$ se tiene
                    que 
                    \[\dv{M}{y} = 3t^{2} = \dv{N}{t}\]
                    Por lo que la ecuación es exacta, esto es que existe una 
                    $\phi$ primitiva de $M$ y $N$\\
                    Luego, integrando $M$ respecto a $t$.
                    \[\int{M dt} = \int{(3t^{2}y+6t)dt} = t^{3}y + 3t^{2} + g(y)\]
                    Integrando $N$ respecto a $y$.
                    \[\int{N dy} = \int{(t^{3}+3y^{2})dt} = t^{3}y + y^{3} + h(t)\]
                    Por lo que la forma de la expresión sin incógnitas es 
                    \[\phi(y, t) = t^{3}y + y^{3} + 3t^{2}\]
                    Luego, despejando $y$
                    \begin{align*}
                        &\implies y = -\frac{y^{3} + 3t^{2}}{t^{3}} \\
                        &\implies y + \frac{y^{3}}{t^{3}} = \frac{3}{t} \\
                        &\implies y + \frac{y^{3}}{t^{3}} - \frac{3}{t}  = 0 \\
                        &\implies y^{3} + yt^{3} - 3t^{2}  = 0
                    \end{align*}
                    Que está en la forma de una ecuación cúbica deprimida que, 
                    por la fórmula de Cardano, tiene solución real
                    \begin{align*}
                        y &= \sqrt[3]{-\frac{-3t^{2}}{2} + \sqrt{\frac{(-3t^{2})^{2}}{4} 
                    + \frac{(t^{3})^{3}}{27}}} + \sqrt[3]{-\frac{-3t^{2}}{2} - \sqrt{\frac{(-3t^{2})^{2}}{4} 
                    + \frac{(t^{3})^{3}}{27}}} \\
                    &= \sqrt[3]{\frac{3t^{2}}{2} + \sqrt{\frac{9t^{4}}{4} 
                    + \frac{t^{9}}{27}}} + \sqrt[3]{\frac{3t^{2}}{2} - \sqrt{\frac{9t^{4}}{4} 
                    + \frac{t^{9}}{27}}}
                    \end{align*}
                }
            \end{enumerate}
        }
        
        % Ejercicio 2
        \item {
            Hallar todas las funciones $g(t)$ que hacen que la ecuación
            diferencial $$y^2\sen{t}+yg(t)\dv{y}{t}=0$$ sea exacta.

            \color{azul}
            % Respuesta
        }
        
        % Ejercicio 3
        \item {
            Las ecuaciones diferenciales de la forma $$\dv{y}{t}=f(y/t)$$ se
            pueden resolver si se hace el cambio de variable $v=y/t$. Mostrar
            que la ecuación toma la forma $$t\dv{v}{t}+v=f(v).$$ Usar este
            método para resolver $$\dv{y}{t}=\frac{t+y}{t-y}.$$

            \color{azul}
            % Respuesta
        }
        
        % Ejercicio 4
        \item {
            Una población crece de acuerdo a la ley logística, y tiene un
            límite de $5\times 10^8$ individuos. Cuando la población es baja se
            suplica cada 40 minutos. ¿Qué valor tendrá la población después de
            dos horas si inicialmente era de a) $10^8$ individuos y b) $10^9$
            individuos?

            \color{azul}
            % Respuesta
        }
    \end{enumerate}
\end{document}